% Options for packages loaded elsewhere
\PassOptionsToPackage{unicode}{hyperref}
\PassOptionsToPackage{hyphens}{url}
\PassOptionsToPackage{dvipsnames,svgnames,x11names}{xcolor}
%
\documentclass[
  letterpaper,
  DIV=11,
  numbers=noendperiod]{scrartcl}

\usepackage{amsmath,amssymb}
\usepackage{iftex}
\ifPDFTeX
  \usepackage[T1]{fontenc}
  \usepackage[utf8]{inputenc}
  \usepackage{textcomp} % provide euro and other symbols
\else % if luatex or xetex
  \usepackage{unicode-math}
  \defaultfontfeatures{Scale=MatchLowercase}
  \defaultfontfeatures[\rmfamily]{Ligatures=TeX,Scale=1}
\fi
\usepackage{lmodern}
\ifPDFTeX\else  
    % xetex/luatex font selection
\fi
% Use upquote if available, for straight quotes in verbatim environments
\IfFileExists{upquote.sty}{\usepackage{upquote}}{}
\IfFileExists{microtype.sty}{% use microtype if available
  \usepackage[]{microtype}
  \UseMicrotypeSet[protrusion]{basicmath} % disable protrusion for tt fonts
}{}
\makeatletter
\@ifundefined{KOMAClassName}{% if non-KOMA class
  \IfFileExists{parskip.sty}{%
    \usepackage{parskip}
  }{% else
    \setlength{\parindent}{0pt}
    \setlength{\parskip}{6pt plus 2pt minus 1pt}}
}{% if KOMA class
  \KOMAoptions{parskip=half}}
\makeatother
\usepackage{xcolor}
\setlength{\emergencystretch}{3em} % prevent overfull lines
\setcounter{secnumdepth}{5}
% Make \paragraph and \subparagraph free-standing
\makeatletter
\ifx\paragraph\undefined\else
  \let\oldparagraph\paragraph
  \renewcommand{\paragraph}{
    \@ifstar
      \xxxParagraphStar
      \xxxParagraphNoStar
  }
  \newcommand{\xxxParagraphStar}[1]{\oldparagraph*{#1}\mbox{}}
  \newcommand{\xxxParagraphNoStar}[1]{\oldparagraph{#1}\mbox{}}
\fi
\ifx\subparagraph\undefined\else
  \let\oldsubparagraph\subparagraph
  \renewcommand{\subparagraph}{
    \@ifstar
      \xxxSubParagraphStar
      \xxxSubParagraphNoStar
  }
  \newcommand{\xxxSubParagraphStar}[1]{\oldsubparagraph*{#1}\mbox{}}
  \newcommand{\xxxSubParagraphNoStar}[1]{\oldsubparagraph{#1}\mbox{}}
\fi
\makeatother

\usepackage{color}
\usepackage{fancyvrb}
\newcommand{\VerbBar}{|}
\newcommand{\VERB}{\Verb[commandchars=\\\{\}]}
\DefineVerbatimEnvironment{Highlighting}{Verbatim}{commandchars=\\\{\}}
% Add ',fontsize=\small' for more characters per line
\usepackage{framed}
\definecolor{shadecolor}{RGB}{241,243,245}
\newenvironment{Shaded}{\begin{snugshade}}{\end{snugshade}}
\newcommand{\AlertTok}[1]{\textcolor[rgb]{0.68,0.00,0.00}{#1}}
\newcommand{\AnnotationTok}[1]{\textcolor[rgb]{0.37,0.37,0.37}{#1}}
\newcommand{\AttributeTok}[1]{\textcolor[rgb]{0.40,0.45,0.13}{#1}}
\newcommand{\BaseNTok}[1]{\textcolor[rgb]{0.68,0.00,0.00}{#1}}
\newcommand{\BuiltInTok}[1]{\textcolor[rgb]{0.00,0.23,0.31}{#1}}
\newcommand{\CharTok}[1]{\textcolor[rgb]{0.13,0.47,0.30}{#1}}
\newcommand{\CommentTok}[1]{\textcolor[rgb]{0.37,0.37,0.37}{#1}}
\newcommand{\CommentVarTok}[1]{\textcolor[rgb]{0.37,0.37,0.37}{\textit{#1}}}
\newcommand{\ConstantTok}[1]{\textcolor[rgb]{0.56,0.35,0.01}{#1}}
\newcommand{\ControlFlowTok}[1]{\textcolor[rgb]{0.00,0.23,0.31}{\textbf{#1}}}
\newcommand{\DataTypeTok}[1]{\textcolor[rgb]{0.68,0.00,0.00}{#1}}
\newcommand{\DecValTok}[1]{\textcolor[rgb]{0.68,0.00,0.00}{#1}}
\newcommand{\DocumentationTok}[1]{\textcolor[rgb]{0.37,0.37,0.37}{\textit{#1}}}
\newcommand{\ErrorTok}[1]{\textcolor[rgb]{0.68,0.00,0.00}{#1}}
\newcommand{\ExtensionTok}[1]{\textcolor[rgb]{0.00,0.23,0.31}{#1}}
\newcommand{\FloatTok}[1]{\textcolor[rgb]{0.68,0.00,0.00}{#1}}
\newcommand{\FunctionTok}[1]{\textcolor[rgb]{0.28,0.35,0.67}{#1}}
\newcommand{\ImportTok}[1]{\textcolor[rgb]{0.00,0.46,0.62}{#1}}
\newcommand{\InformationTok}[1]{\textcolor[rgb]{0.37,0.37,0.37}{#1}}
\newcommand{\KeywordTok}[1]{\textcolor[rgb]{0.00,0.23,0.31}{\textbf{#1}}}
\newcommand{\NormalTok}[1]{\textcolor[rgb]{0.00,0.23,0.31}{#1}}
\newcommand{\OperatorTok}[1]{\textcolor[rgb]{0.37,0.37,0.37}{#1}}
\newcommand{\OtherTok}[1]{\textcolor[rgb]{0.00,0.23,0.31}{#1}}
\newcommand{\PreprocessorTok}[1]{\textcolor[rgb]{0.68,0.00,0.00}{#1}}
\newcommand{\RegionMarkerTok}[1]{\textcolor[rgb]{0.00,0.23,0.31}{#1}}
\newcommand{\SpecialCharTok}[1]{\textcolor[rgb]{0.37,0.37,0.37}{#1}}
\newcommand{\SpecialStringTok}[1]{\textcolor[rgb]{0.13,0.47,0.30}{#1}}
\newcommand{\StringTok}[1]{\textcolor[rgb]{0.13,0.47,0.30}{#1}}
\newcommand{\VariableTok}[1]{\textcolor[rgb]{0.07,0.07,0.07}{#1}}
\newcommand{\VerbatimStringTok}[1]{\textcolor[rgb]{0.13,0.47,0.30}{#1}}
\newcommand{\WarningTok}[1]{\textcolor[rgb]{0.37,0.37,0.37}{\textit{#1}}}

\providecommand{\tightlist}{%
  \setlength{\itemsep}{0pt}\setlength{\parskip}{0pt}}\usepackage{longtable,booktabs,array}
\usepackage{calc} % for calculating minipage widths
% Correct order of tables after \paragraph or \subparagraph
\usepackage{etoolbox}
\makeatletter
\patchcmd\longtable{\par}{\if@noskipsec\mbox{}\fi\par}{}{}
\makeatother
% Allow footnotes in longtable head/foot
\IfFileExists{footnotehyper.sty}{\usepackage{footnotehyper}}{\usepackage{footnote}}
\makesavenoteenv{longtable}
\usepackage{graphicx}
\makeatletter
\newsavebox\pandoc@box
\newcommand*\pandocbounded[1]{% scales image to fit in text height/width
  \sbox\pandoc@box{#1}%
  \Gscale@div\@tempa{\textheight}{\dimexpr\ht\pandoc@box+\dp\pandoc@box\relax}%
  \Gscale@div\@tempb{\linewidth}{\wd\pandoc@box}%
  \ifdim\@tempb\p@<\@tempa\p@\let\@tempa\@tempb\fi% select the smaller of both
  \ifdim\@tempa\p@<\p@\scalebox{\@tempa}{\usebox\pandoc@box}%
  \else\usebox{\pandoc@box}%
  \fi%
}
% Set default figure placement to htbp
\def\fps@figure{htbp}
\makeatother

\usepackage{booktabs}
\usepackage{longtable}
\usepackage{array}
\usepackage{multirow}
\usepackage{wrapfig}
\usepackage{float}
\usepackage{colortbl}
\usepackage{pdflscape}
\usepackage{tabu}
\usepackage{threeparttable}
\usepackage{threeparttablex}
\usepackage[normalem]{ulem}
\usepackage{makecell}
\usepackage{xcolor}
\KOMAoption{captions}{tableheading}
\makeatletter
\@ifpackageloaded{caption}{}{\usepackage{caption}}
\AtBeginDocument{%
\ifdefined\contentsname
  \renewcommand*\contentsname{Table of contents}
\else
  \newcommand\contentsname{Table of contents}
\fi
\ifdefined\listfigurename
  \renewcommand*\listfigurename{List of Figures}
\else
  \newcommand\listfigurename{List of Figures}
\fi
\ifdefined\listtablename
  \renewcommand*\listtablename{List of Tables}
\else
  \newcommand\listtablename{List of Tables}
\fi
\ifdefined\figurename
  \renewcommand*\figurename{Figure}
\else
  \newcommand\figurename{Figure}
\fi
\ifdefined\tablename
  \renewcommand*\tablename{Table}
\else
  \newcommand\tablename{Table}
\fi
}
\@ifpackageloaded{float}{}{\usepackage{float}}
\floatstyle{ruled}
\@ifundefined{c@chapter}{\newfloat{codelisting}{h}{lop}}{\newfloat{codelisting}{h}{lop}[chapter]}
\floatname{codelisting}{Listing}
\newcommand*\listoflistings{\listof{codelisting}{List of Listings}}
\makeatother
\makeatletter
\makeatother
\makeatletter
\@ifpackageloaded{caption}{}{\usepackage{caption}}
\@ifpackageloaded{subcaption}{}{\usepackage{subcaption}}
\makeatother

\usepackage{bookmark}

\IfFileExists{xurl.sty}{\usepackage{xurl}}{} % add URL line breaks if available
\urlstyle{same} % disable monospaced font for URLs
\hypersetup{
  pdftitle={HUM.A.N OPS -- Relatório},
  pdfauthor={Seu Grupo},
  colorlinks=true,
  linkcolor={blue},
  filecolor={Maroon},
  citecolor={Blue},
  urlcolor={Blue},
  pdfcreator={LaTeX via pandoc}}


\title{HUM.A.N OPS -- Relatório}
\author{Seu Grupo}
\date{}

\begin{document}
\maketitle

\renewcommand*\contentsname{Table of contents}
{
\hypersetup{linkcolor=}
\setcounter{tocdepth}{3}
\tableofcontents
}

\section{1. Introdução}\label{introduuxe7uxe3o}

Este relatório apresenta a análise integrada das três frentes do HUM.A.N
OPS:

\begin{itemize}
\item
  \textbf{Bem-estar} (Hygeia)
\item
  \textbf{Inclusão e Fairness} (Sophia)
\item
  \textbf{Sustentabilidade e Energia} (Gaia)
\end{itemize}

Ele incorpora:

\begin{itemize}
\item
  análises temporais
\item
  gráficos avançados
\item
  KPIs
\item
  tabelas formatadas profissionalmente
\end{itemize}

\section{2. Bem-Estar}\label{bem-estar}

2.1. Base de dados

\begin{Shaded}
\begin{Highlighting}[]
\NormalTok{checkin }\OtherTok{\textless{}{-}} \FunctionTok{dbReadTable}\NormalTok{(con, }\StringTok{"checkin"}\NormalTok{)}
\NormalTok{colaboradores }\OtherTok{\textless{}{-}} \FunctionTok{dbReadTable}\NormalTok{(con, }\StringTok{"colaborador"}\NormalTok{)}

\NormalTok{df }\OtherTok{\textless{}{-}}\NormalTok{ checkin }\SpecialCharTok{|\textgreater{}}
\FunctionTok{inner\_join}\NormalTok{(colaboradores, }\AttributeTok{by =} \FunctionTok{c}\NormalTok{(}\StringTok{"id\_colab"} \OtherTok{=} \StringTok{"id\_colab"}\NormalTok{)) }\SpecialCharTok{|\textgreater{}}
\FunctionTok{mutate}\NormalTok{(}
\AttributeTok{dt =} \FunctionTok{as.POSIXct}\NormalTok{(dt),}
\AttributeTok{nome =}\NormalTok{ nm\_colaborador}
\NormalTok{)}
\end{Highlighting}
\end{Shaded}

2.2. Evolução dos indicadores

\begin{Shaded}
\begin{Highlighting}[]
\NormalTok{df\_long }\OtherTok{\textless{}{-}}\NormalTok{ df }\SpecialCharTok{|\textgreater{}}
\FunctionTok{select}\NormalTok{(dt, nome, q1, q2, q3) }\SpecialCharTok{|\textgreater{}}
\FunctionTok{rename}\NormalTok{(}
\AttributeTok{motivacao =}\NormalTok{ q1,}
\AttributeTok{cansaco =}\NormalTok{ q2,}
\AttributeTok{stress =}\NormalTok{ q3}
\NormalTok{) }\SpecialCharTok{|\textgreater{}}
\NormalTok{tidyr}\SpecialCharTok{::}\FunctionTok{pivot\_longer}\NormalTok{(}
\AttributeTok{cols =} \FunctionTok{c}\NormalTok{(motivacao, cansaco, stress),}
\AttributeTok{names\_to =} \StringTok{"variavel"}\NormalTok{,}
\AttributeTok{values\_to =} \StringTok{"valor"}
\NormalTok{)}

\FunctionTok{ggplot}\NormalTok{(df\_long, }\FunctionTok{aes}\NormalTok{(}\AttributeTok{x =}\NormalTok{ dt, }\AttributeTok{y =}\NormalTok{ valor, }\AttributeTok{color =}\NormalTok{ variavel)) }\SpecialCharTok{+}
\FunctionTok{geom\_line}\NormalTok{(}\AttributeTok{size =} \DecValTok{1}\NormalTok{) }\SpecialCharTok{+}
\FunctionTok{geom\_smooth}\NormalTok{(}\AttributeTok{se =} \ConstantTok{FALSE}\NormalTok{, }\AttributeTok{linewidth =} \FloatTok{1.1}\NormalTok{) }\SpecialCharTok{+}
\FunctionTok{scale\_color\_manual}\NormalTok{(}\AttributeTok{values =}\NormalTok{ cores) }\SpecialCharTok{+}
\FunctionTok{labs}\NormalTok{(}
\AttributeTok{title =} \StringTok{"Evolução dos Indicadores de Bem{-}Estar"}\NormalTok{,}
\AttributeTok{x =} \StringTok{"Data"}\NormalTok{,}
\AttributeTok{y =} \StringTok{"Intensidade (1–5)"}\NormalTok{,}
\AttributeTok{color =} \StringTok{"Variável"}
\NormalTok{) }\SpecialCharTok{+}
\FunctionTok{theme\_minimal}\NormalTok{(}\AttributeTok{base\_size =} \DecValTok{13}\NormalTok{)}
\end{Highlighting}
\end{Shaded}

\begin{verbatim}
Warning: Using `size` aesthetic for lines was deprecated in ggplot2 3.4.0.
i Please use `linewidth` instead.
\end{verbatim}

\begin{verbatim}
`geom_smooth()` using method = 'loess' and formula = 'y ~ x'
\end{verbatim}

\pandocbounded{\includegraphics[keepaspectratio]{human_ops_report_files/figure-pdf/unnamed-chunk-3-1.pdf}}

2.3. Stress por colaborador (facet)

\begin{Shaded}
\begin{Highlighting}[]
\FunctionTok{ggplot}\NormalTok{(df, }\FunctionTok{aes}\NormalTok{(}\AttributeTok{x =}\NormalTok{ dt, }\AttributeTok{y =}\NormalTok{ q3, }\AttributeTok{color =}\NormalTok{ nome)) }\SpecialCharTok{+}
\FunctionTok{geom\_line}\NormalTok{(}\AttributeTok{size =} \DecValTok{1}\NormalTok{) }\SpecialCharTok{+}
\FunctionTok{facet\_wrap}\NormalTok{(}\SpecialCharTok{\textasciitilde{}}\NormalTok{nome, }\AttributeTok{scales =} \StringTok{"free\_y"}\NormalTok{) }\SpecialCharTok{+}
\FunctionTok{labs}\NormalTok{(}
\AttributeTok{title =} \StringTok{"Stress por Colaborador"}\NormalTok{,}
\AttributeTok{x =} \StringTok{"Data"}\NormalTok{,}
\AttributeTok{y =} \StringTok{"Stress (1{-}5)"}
\NormalTok{) }\SpecialCharTok{+}
\FunctionTok{theme\_minimal}\NormalTok{(}\AttributeTok{base\_size =} \DecValTok{12}\NormalTok{)}
\end{Highlighting}
\end{Shaded}

\begin{verbatim}
`geom_line()`: Each group consists of only one observation.
i Do you need to adjust the group aesthetic?
`geom_line()`: Each group consists of only one observation.
i Do you need to adjust the group aesthetic?
`geom_line()`: Each group consists of only one observation.
i Do you need to adjust the group aesthetic?
\end{verbatim}

\pandocbounded{\includegraphics[keepaspectratio]{human_ops_report_files/figure-pdf/unnamed-chunk-4-1.pdf}}

2.4. KPI -- Médias gerais

\begin{Shaded}
\begin{Highlighting}[]
\NormalTok{kpis }\OtherTok{\textless{}{-}}\NormalTok{ df\_long }\SpecialCharTok{\%\textgreater{}\%}
  \FunctionTok{group\_by}\NormalTok{(variavel) }\SpecialCharTok{\%\textgreater{}\%}
  \FunctionTok{summarise}\NormalTok{(}\AttributeTok{media =} \FunctionTok{round}\NormalTok{(}\FunctionTok{mean}\NormalTok{(valor), }\DecValTok{2}\NormalTok{))}

\NormalTok{kpis }\SpecialCharTok{\%\textgreater{}\%}
  \FunctionTok{kable}\NormalTok{() }\SpecialCharTok{\%\textgreater{}\%}
\NormalTok{  kableExtra}\SpecialCharTok{::}\FunctionTok{kable\_styling}\NormalTok{(}\AttributeTok{full\_width =} \ConstantTok{FALSE}\NormalTok{, }\AttributeTok{bootstrap\_options =} \StringTok{"striped"}\NormalTok{)}
\end{Highlighting}
\end{Shaded}

\begin{longtable*}[t]{lr}
\toprule
variavel & media\\
\midrule
cansaco & 2.84\\
motivacao & 3.05\\
stress & 2.89\\
\bottomrule
\end{longtable*}

\section{3. Inclusão (Fairness)}\label{inclusuxe3o-fairness}

3.1. Base

\begin{Shaded}
\begin{Highlighting}[]
\NormalTok{inclusao }\OtherTok{\textless{}{-}} \FunctionTok{dbReadTable}\NormalTok{(con, }\StringTok{"inclusao\_recrut"}\NormalTok{) }\SpecialCharTok{|\textgreater{}}
\FunctionTok{mutate}\NormalTok{(}
\AttributeTok{aprovado =} \FunctionTok{as.numeric}\NormalTok{(aprovado),}
\AttributeTok{grupo\_ref =} \FunctionTok{as.factor}\NormalTok{(grupo\_ref)}
\NormalTok{)}
\end{Highlighting}
\end{Shaded}

3.2 Taxa de aprovação por grupo

\begin{Shaded}
\begin{Highlighting}[]
\NormalTok{tab }\OtherTok{\textless{}{-}}\NormalTok{ inclusao }\SpecialCharTok{|\textgreater{}}
\FunctionTok{group\_by}\NormalTok{(grupo\_ref) }\SpecialCharTok{|\textgreater{}}
\FunctionTok{summarise}\NormalTok{(}
\AttributeTok{taxa\_aprovacao =} \FunctionTok{mean}\NormalTok{(aprovado),}
\AttributeTok{candidatos =} \FunctionTok{n}\NormalTok{()}
\NormalTok{)}

\NormalTok{tab }\SpecialCharTok{|\textgreater{}}
\FunctionTok{kable}\NormalTok{(}\AttributeTok{digits =} \DecValTok{3}\NormalTok{) }\SpecialCharTok{|\textgreater{}}
\FunctionTok{kable\_styling}\NormalTok{(}\AttributeTok{full\_width =} \ConstantTok{FALSE}\NormalTok{, }\AttributeTok{bootstrap\_options =} \StringTok{"striped"}\NormalTok{)}
\end{Highlighting}
\end{Shaded}

\begin{longtable*}[t]{lrr}
\toprule
grupo\_ref & taxa\_aprovacao & candidatos\\
\midrule
Grupo A & 0.500 & 2\\
Grupo B & 0.333 & 3\\
\bottomrule
\end{longtable*}

3.3. Gráfico -- Fairness

\begin{Shaded}
\begin{Highlighting}[]
\FunctionTok{ggplot}\NormalTok{(tab, }\FunctionTok{aes}\NormalTok{(}\AttributeTok{x =}\NormalTok{ grupo\_ref, }\AttributeTok{y =}\NormalTok{ taxa\_aprovacao, }\AttributeTok{fill =}\NormalTok{ grupo\_ref)) }\SpecialCharTok{+}
\FunctionTok{geom\_col}\NormalTok{() }\SpecialCharTok{+}
\FunctionTok{scale\_fill\_manual}\NormalTok{(}\AttributeTok{values =}\NormalTok{ cores[}\StringTok{"inclusao"}\NormalTok{]) }\SpecialCharTok{+}
\FunctionTok{scale\_y\_continuous}\NormalTok{(}\AttributeTok{labels =} \FunctionTok{percent\_format}\NormalTok{()) }\SpecialCharTok{+}
\FunctionTok{labs}\NormalTok{(}
\AttributeTok{title =} \StringTok{"Taxa de Aprovação por Grupo"}\NormalTok{,}
\AttributeTok{x =} \StringTok{"Grupo"}\NormalTok{,}
\AttributeTok{y =} \StringTok{"Aprovação (\%)"}
\NormalTok{) }\SpecialCharTok{+}
\FunctionTok{theme\_minimal}\NormalTok{(}\AttributeTok{base\_size =} \DecValTok{13}\NormalTok{)}
\end{Highlighting}
\end{Shaded}

\begin{verbatim}
Warning: No shared levels found between `names(values)` of the manual scale and the
data's fill values.
No shared levels found between `names(values)` of the manual scale and the
data's fill values.
\end{verbatim}

\pandocbounded{\includegraphics[keepaspectratio]{human_ops_report_files/figure-pdf/unnamed-chunk-8-1.pdf}}

\section{4. Sustentabilidade (Energia)}\label{sustentabilidade-energia}

4.1. Base

\begin{Shaded}
\begin{Highlighting}[]
\NormalTok{energia }\OtherTok{\textless{}{-}} \FunctionTok{dbReadTable}\NormalTok{(con, }\StringTok{"energia"}\NormalTok{) }\SpecialCharTok{|\textgreater{}}
\FunctionTok{mutate}\NormalTok{(}\AttributeTok{dt =} \FunctionTok{as.POSIXct}\NormalTok{(dt))}
\end{Highlighting}
\end{Shaded}

4.2. Gráfico temporal

\begin{Shaded}
\begin{Highlighting}[]
\FunctionTok{ggplot}\NormalTok{(energia, }\FunctionTok{aes}\NormalTok{(}\AttributeTok{x =}\NormalTok{ dt, }\AttributeTok{y =}\NormalTok{ kwh)) }\SpecialCharTok{+}
\FunctionTok{geom\_line}\NormalTok{(}\AttributeTok{color =}\NormalTok{ cores[}\StringTok{"energia"}\NormalTok{], }\AttributeTok{linewidth =} \DecValTok{1}\NormalTok{) }\SpecialCharTok{+}
\FunctionTok{geom\_smooth}\NormalTok{(}\AttributeTok{color =} \StringTok{"black"}\NormalTok{, }\AttributeTok{linewidth =} \DecValTok{1}\NormalTok{, }\AttributeTok{se =} \ConstantTok{FALSE}\NormalTok{) }\SpecialCharTok{+}
\FunctionTok{labs}\NormalTok{(}
\AttributeTok{title =} \StringTok{"Consumo de Energia ao Longo do Tempo"}\NormalTok{,}
\AttributeTok{x =} \StringTok{"Data"}\NormalTok{,}
\AttributeTok{y =} \StringTok{"kWh"}
\NormalTok{) }\SpecialCharTok{+}
\FunctionTok{theme\_minimal}\NormalTok{(}\AttributeTok{base\_size =} \DecValTok{13}\NormalTok{)}
\end{Highlighting}
\end{Shaded}

\begin{verbatim}
`geom_smooth()` using method = 'loess' and formula = 'y ~ x'
\end{verbatim}

\pandocbounded{\includegraphics[keepaspectratio]{human_ops_report_files/figure-pdf/unnamed-chunk-10-1.pdf}}

4.3. Heatmap horário (se houver granularidade)

\begin{Shaded}
\begin{Highlighting}[]
\NormalTok{energia\_hm }\OtherTok{\textless{}{-}}\NormalTok{ energia }\SpecialCharTok{|\textgreater{}}
\FunctionTok{mutate}\NormalTok{(}
\AttributeTok{dia =} \FunctionTok{as.Date}\NormalTok{(dt),}
\AttributeTok{hora =} \FunctionTok{format}\NormalTok{(dt, }\StringTok{"\%H"}\NormalTok{)}
\NormalTok{) }\SpecialCharTok{|\textgreater{}}
\FunctionTok{group\_by}\NormalTok{(dia, hora) }\SpecialCharTok{|\textgreater{}}
\FunctionTok{summarise}\NormalTok{(}\AttributeTok{kwh\_medio =} \FunctionTok{mean}\NormalTok{(kwh))}
\end{Highlighting}
\end{Shaded}

\begin{verbatim}
`summarise()` has grouped output by 'dia'. You can override using the `.groups`
argument.
\end{verbatim}

\begin{Shaded}
\begin{Highlighting}[]
\FunctionTok{ggplot}\NormalTok{(energia\_hm, }\FunctionTok{aes}\NormalTok{(}\AttributeTok{x =}\NormalTok{ hora, }\AttributeTok{y =}\NormalTok{ dia, }\AttributeTok{fill =}\NormalTok{ kwh\_medio)) }\SpecialCharTok{+}
\FunctionTok{geom\_tile}\NormalTok{() }\SpecialCharTok{+}
\FunctionTok{scale\_fill\_gradient}\NormalTok{(}\AttributeTok{low =} \StringTok{"white"}\NormalTok{, }\AttributeTok{high =}\NormalTok{ cores[}\StringTok{"energia"}\NormalTok{]) }\SpecialCharTok{+}
\FunctionTok{labs}\NormalTok{(}\AttributeTok{title =} \StringTok{"Heatmap de Consumo de Energia"}\NormalTok{, }\AttributeTok{x =} \StringTok{"Hora"}\NormalTok{, }\AttributeTok{y =} \StringTok{"Dia"}\NormalTok{) }\SpecialCharTok{+}
\FunctionTok{theme\_minimal}\NormalTok{(}\AttributeTok{base\_size =} \DecValTok{12}\NormalTok{)}
\end{Highlighting}
\end{Shaded}

\pandocbounded{\includegraphics[keepaspectratio]{human_ops_report_files/figure-pdf/unnamed-chunk-11-1.pdf}}

\section{5. Conclusões}\label{conclusuxf5es}

\begin{itemize}
\item
  \textbf{Bem-estar:} tendências visuais ajudam a identificar riscos de
  burnout.
\item
  \textbf{Inclusão:} fairness pode ser monitorado ao longo do processo
  de recrutamento.
\end{itemize}

\begin{itemize}
\tightlist
\item
  \textbf{Energia:} padrões e anomalias podem ser vistos claramente nos
  gráficos
\end{itemize}

\begin{Shaded}
\begin{Highlighting}[]
\FunctionTok{dbDisconnect}\NormalTok{(con)}
\end{Highlighting}
\end{Shaded}





\end{document}
